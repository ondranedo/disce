\documentclass[a4paper,12pt]{report}
\usepackage[utf8]{inputenc}
\usepackage[czech]{babel}
\usepackage{geometry}
\usepackage[table,xcdraw]{xcolor}
\usepackage{colortbl}
\geometry{margin=2cm}

\begin{document}

\begin{titlepage}
    \centering
    
    {\scshape\Large Lambda Science, s.r.o. \par}
    \vspace{2cm}
    
    {\huge\bfseries Podnikatelský Záměr\par}
    \vspace{7cm}
    
    {\Large\bf 
    ZAHÁJENÍ PODNIKATELSKÉ ČINNOSTI, V RÁMCI RENOMOVÁNÍ MONUMENTÁLNÍCH TECHNOLOGICKÝCH A 
    STÁTNÍCH INSTITUTŮ, S UČELY NEJEN MAXIMALIZAČNĚ VÝDĚLEČNÝMI ČI SRPÁVNÍMI, UČINĚNO NAHRAZENÍM KOGNITIVITY ŘEČENÝCH INSTITUTŮ\par}
    \vfill
    

    {\large Prosinec, 2023\par}
\end{titlepage}

\chapter{Preambule}
Nacházíme se na pokraji éry, kdy každý den, hodina a minuta nás směřují k 
nevyhnutelné revoluci. Tato revoluce, konečná a v pesimistickém světle bohužel 
opodstatnitelná a fatální, oznamuje proměnu. Kriticky nutné je nyní začít připravovat
 ekonomickou a sociální strukturu států, či spíše celé společnosti na tuto 
 zásadní změnu. Je to výzva, která vyžaduje nejen adaptaci, ale i aktivní 
 utváření nového směru pro budoucnost.


\tableofcontents


\chapter{Podnikateslý záměr}
\section{Vstupní profil}
\subsection{Charakterizování subjektu}
\subsubsection{Název firmy}
Lambda Science, s.r.o.
\subsubsection{Sídlo}
Praha 2, Karlovo náměstí 12
\subsubsection{Identifikační číslo}
Pude přiděleno
\subsubsection{Plátce DPH}
Ano
\subsubsection{Datum zahájení činnosti podnikání}
dne 29. února 2024

\subsubsection{Obor podnikání}                                
Lambda Science, s.r.o., se bude zaměřovat na poskytování komplexního poradenství pro malé, střední a velké firmy,
 stejně jako pro státní instituce včetně zdravotních ústavů, finančních úřadů, škol a České národní banky. Naše 
 specializace bude spočívat v implementaci pokročilých jazykových modelů\footnote{Jazykový model jest forma umělé inteligence} či jiných, s cílem optimalizovat a inovovat každodenní operace.

V rámci naší činnosti se budeme věnovat analyzování a optimalizaci firemních procesů, zejména v oblastech, 
které mohou být zefektivněny a zautomatizovány pomocí moderních technologií. Naše služby budou zahrnovat návrh a implementaci 
jazykových modelů, školení zaměstnanců pro práci s novými technologiemi, dlouhodobou správu implementovaných softwarových řešení
a pravidelné aktualizace.

Dále budeme rozvíjet spolupráce se zahraničními a tuzemskými univerzitami a firmami v oblasti výzkumu a financování vývoje. 

\subsection{Profesní a osobní údaje zakladatelů}

Jakož jediný společník a zakladatel firmy je pan Ondřej Nedojedlý, narozen 27. listopadu 2004, bytem Tovární 2447/4, 735 06 Karviná.

Pan Nedojedlý je absolventem Střední průmyslové školy elektrotechnické, obor Informační technologie v Havířově. Již během studia 
se pan Nedojedlý vášnivě zabýval nejrůznějším vývojem v oblasti umělé inteligence, nových technologií a modernizace.

Jeho schopnosti zahrnují tvorbu a implementaci pokročilých jazykových modelů, programování a široký záběr dovedností v oblasti
 informačních technologií. S výrazným nadšením a vizionářským přístupem má pan Nedojedlý zájem o inovativní řešení a efektivní
  využití moderních technologií ve prospěch budoucích klientů Lambda Science, s.r.o.

  V oblasti komunikace a mezilidských dovedností se pan Nedojedlý vyznačuje schopností efektivně sdělovat složité technické koncepty 
  laikům a pracovat v týmovém prostředí. Jeho zkušenosti zahrnují vedoucí role v projektech, kde bylo zapotřebí nejen technického
   know-how, ale i schopnosti efektivního jednání s týmem.

\section{Založení firmy}
\subsection{Právní forma}

Společnost Lambda Science, s.r.o. je společností s ručením omezeným, která vznikla zápisem 
do obchodního rejstříku vedeného Krajským soudem v Ostravě dne 10. listopdadu, 2023.
Jediným společníkem a zároveň jediným jednatelem firmy je pan Ondřej Nedojedlý, narozen 27. 11. 2004, bytem Tovární 2447/4, 735 06 Karviná.

\subsection{Velikost základního kapitálu}
Výše základního kapitálu je 1 mil. Kč a základní kapitál je celý splacen.

\newpage

\section{Marketing}

\subsection{Průzkum trhu}

Na základě průzkumu trhu byly identifikovány klíčová odvětví; jelikož konkurence v těcho odvětví neexistuje,
vidíme obrovský potenciál pro růst firmy. Odvětví, které může firma nahradit, mají nevyslovitelnou tržní kapacitu.

\subsubsection{Odvětví působnosti}

\begin{enumerate}
    \item kontorle kamerových systému,
    \item technické správy budovy,
    \item organizace časového rozvrhu,
    \item provádění objednávek zboží,
    \item technické poradentství zaměstnancům,
    \item hledání účetních chyb,
    \item nabíraní nových zaměstnanců,
    \item řešení sporů na pracovišti,
    \item nahrazení IT oddělení,
    \item kontrolování práce zaměstnanců,
    \item optimalizace databází firmy,
    \item zákaznické linka,
    \item zabezpečení proti Kybernetický útok,
    \item strategický rozvoj firmy,
    \item zaškolování nových zaměstnanců,
    \item \dots
\end{enumerate}

Ve všech výše řečených odvětví je možno implementovat umělou inteligenci.

\newpage

\subsection{Marketingový mix}
\subsubsection{Produkt}
Firma bude z prvu budovat jméno a značku, a tak se bude zaměřovat na menší podniky a konzervativnější implementace.
V první fázy bude firma nabízet produkt: \textbf{AI CO-MNIGLME} - Bude pomáhat s během firmy, umělá inteligence bude
určitá "opěra" pro majitelé firem. Proto bude uměla inteligence fungovat pro obecné užití, bude moct kontrolovat
dokumenty, odpovídat na e-maily, posílat e-maly, nastavovat termostat, klimatizaci, objednávat zboží, dle statistik 
které vypozoruje, provádět personální doporučení, pročítat životopisy, motivační dopisy uchazečů, a navrhnout majiteli
vhodného kandidáta\dots Cena produktu bude řešena domluvou, jelikož nelze určit obecnou cenu, bude však vybírána měsíčním
paušálem, jež bude stanoven. 

Po vybudování jména firmy a reputace na trhu, se bude nabízer produkt: \textbf{AI RE-PLACE} - Produkt již nebude zaměřován
na více odvětví, bude plnit jen jedno, ale za to lépe než kterýkoliv zaměstnanec. Například bude nabízen produkt
\textbf{AI RE-PLACE TIMETABLE}, kde umělá inteligence bude schopna perfektvě navrhovat směny zaměstnanců, konferenční schůzky,
zaměstanci budou moct umělé inteligence dávat požadavky (dovolené, rezervace konfrerenčních místností, schůzky s řediteli, ...),
které budou schváleny, či nikoliv. Tyto specifické balíčky budou řešeny kvartálním paušálem, kde se cena bude opět určovat od
konkrétních případů. Cena se však bude pohybovat v násobně vyšších řádech než-li produkt \textbf{AI CO-MNIGLME}.

\subsubsection{Cena}

Cena bude určována hodinově v aktivitě umělé inteligence, kterou si firma bude volit, a hrubým odhadem bude stanoven měsíční/kvartální paušál.
Pokud bude firma či instituce vyžadovat AI CO-MINGLE, bude zpočítána měsíční aktivita pro fungováni 10h denně: 

    $$ 10hodin * 5dnu  * 4tydny = 200skalar$$

Náročnost provozu bude stanovena od 100 výše, dle domluvy. A výsledná cena bude:

    $$ 200sklar * 111 narocnost = 22220kc/mesic$$

\subsubsection{Propagace}

Propagace bude prováděna přes osobní návštěvy firem a přednášky s demonstracemi, osobní návštěvy sjezdů podnikatelů, přednášky na
České podnikatelské alianci, či jiných sdružení.

\subsubsection{Distribuce}

Distrubování produktů bude prováděno osobní instalací a zaškolením, po domluvě se zákaznickou firmou.

\newpage

\subsection{SWOT Analýza}

\begin{table}[h]
    \begin{tabular}{|
    >{\columncolor[HTML]{FFCCC9}}l |
    >{\columncolor[HTML]{9AFF99}}l |}
    \hline
    {\color[HTML]{000000} \textbf{Silné stránky}}                                                                                                   & {\color[HTML]{000000} \textbf{Slabé stránky}}                                                                                            \\
    {\color[HTML]{000000} \begin{tabular}[c]{@{}l@{}}- žádná konkurence\\ - potenciál růstu\end{tabular}}                                           & {\color[HTML]{000000} \begin{tabular}[c]{@{}l@{}}- nulové jméno na trhu\\ - strach z nových technologií\end{tabular}}                    \\ \hline
    \cellcolor[HTML]{CBCEFB}{\color[HTML]{000000} \textbf{Příležitosti}}                                                                            & \cellcolor[HTML]{FFFC9E}{\color[HTML]{000000} \textbf{Ohrožení}}                                                                         \\
    \cellcolor[HTML]{CBCEFB}{\color[HTML]{000000} \begin{tabular}[c]{@{}l@{}}- angažování ve státním sektru\\ - revitalizace školství\end{tabular}} & \cellcolor[HTML]{FFFC9E}{\color[HTML]{000000} \begin{tabular}[c]{@{}l@{}}- nenalezení zákaznické firmy\\ - prostety odborů\end{tabular}} \\ \hline
    \end{tabular}
    \end{table}


\section{Organizační a personální zajištění}

\newpage
\section{Finanční plán}

\newpage
\section{Závěr}

\end{document}
